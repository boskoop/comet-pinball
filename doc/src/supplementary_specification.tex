%   Copyright 2012 Comet Engineering, Patrick Haring & Christian Bürgi
%
%   Licensed under the Apache License, Version 2.0 (the "License");
%   you may not use this file except in compliance with the License.
%   You may obtain a copy of the License at
%
%       http://www.apache.org/licenses/LICENSE-2.0
%
%   Unless required by applicable law or agreed to in writing, software
%   distributed under the License is distributed on an "AS IS" BASIS,
%   WITHOUT WARRANTIES OR CONDITIONS OF ANY KIND, either express or implied.
%   See the License for the specific language governing permissions and
%   limitations under the License.

\documentclass[fontsize=12pt,
               paper=a4,
               twoside=false,
               parskip=half,
               ]{scrartcl}

% Load the packages
%   Copyright 2012 Comet Engineering, Patrick Haring & Christian Bürgi
%
%   Licensed under the Apache License, Version 2.0 (the "License");
%   you may not use this file except in compliance with the License.
%   You may obtain a copy of the License at
%
%       http://www.apache.org/licenses/LICENSE-2.0
%
%   Unless required by applicable law or agreed to in writing, software
%   distributed under the License is distributed on an "AS IS" BASIS,
%   WITHOUT WARRANTIES OR CONDITIONS OF ANY KIND, either express or implied.
%   See the License for the specific language governing permissions and
%   limitations under the License.

% Packages Template
% =================
% 
% Contains packages used for project documentation
% 
% @author burgc5
% 
% To use this simply enter: %   Copyright 2012 Comet Engineering, Patrick Haring & Christian Bürgi
%
%   Licensed under the Apache License, Version 2.0 (the "License");
%   you may not use this file except in compliance with the License.
%   You may obtain a copy of the License at
%
%       http://www.apache.org/licenses/LICENSE-2.0
%
%   Unless required by applicable law or agreed to in writing, software
%   distributed under the License is distributed on an "AS IS" BASIS,
%   WITHOUT WARRANTIES OR CONDITIONS OF ANY KIND, either express or implied.
%   See the License for the specific language governing permissions and
%   limitations under the License.

% Packages Template
% =================
% 
% Contains packages used for project documentation
% 
% @author burgc5
% 
% To use this simply enter: %   Copyright 2012 Comet Engineering, Patrick Haring & Christian Bürgi
%
%   Licensed under the Apache License, Version 2.0 (the "License");
%   you may not use this file except in compliance with the License.
%   You may obtain a copy of the License at
%
%       http://www.apache.org/licenses/LICENSE-2.0
%
%   Unless required by applicable law or agreed to in writing, software
%   distributed under the License is distributed on an "AS IS" BASIS,
%   WITHOUT WARRANTIES OR CONDITIONS OF ANY KIND, either express or implied.
%   See the License for the specific language governing permissions and
%   limitations under the License.

% Packages Template
% =================
% 
% Contains packages used for project documentation
% 
% @author burgc5
% 
% To use this simply enter: \input{./packages.tex}

\usepackage[utf8]{inputenc}
\usepackage[T1]{fontenc}

% Set font to latin modern
\usepackage{lmodern}

\usepackage[pdftex]{graphicx}
\usepackage{float}
\usepackage{epstopdf}
\usepackage{caption}
\usepackage{subcaption}

% Nice source code listings
\usepackage[usenames,dvipsnames]{xcolor}
\usepackage{listings}
\DeclareCaptionFont{white}{\color{white}}
\DeclareCaptionFormat{listing}{\colorbox{gray}{\parbox{\textwidth}{#1#2#3}}}
\captionsetup[lstlisting]{format=listing,labelfont=white,textfont=white}
\lstset{ %
  backgroundcolor=\color{white},  % choose the background color; you must add \usepackage{color} or \usepackage{xcolor}
  basicstyle=\footnotesize,       % the size of the fonts that are used for the code
  breakatwhitespace=false,        % sets if automatic breaks should only happen at whitespace
  breaklines=true,                % sets automatic line breaking
  numbers=none,                   % where to put the line-numbers; possible values are (none, left, right)
  showspaces=false,               % show spaces everywhere adding particular underscores; it overrides 'showstringspaces'
  showstringspaces=false,         % underline spaces within strings only
  showtabs=false,                 % show tabs within strings adding particular underscores
  tabsize=2,                      % sets default tabsize to 2 spaces
}

% Create links in pdf documents
\usepackage[colorlinks,pdfpagelabels,pdfstartview=FitH,bookmarksopen=true,bookmarksnumbered=true,linkcolor=black,plainpages=false,hypertexnames=false,citecolor=black] {hyperref}
\hypersetup{
    colorlinks,%
    citecolor=black,%
    filecolor=black,%
    linkcolor=black,%
    urlcolor=black
}
\urlstyle{same}

% Use \enquote{} to create quotation marks
\usepackage{csquotes}

% Create professional tables with booktabs
% @see http://en.wikibooks.org/wiki/LaTeX/Tables#Professional_tables
\usepackage{booktabs}

% Customizable enumerates/itemizes
\usepackage{enumitem}

% git meta information
\usepackage{gitinfo}


\usepackage[utf8]{inputenc}
\usepackage[T1]{fontenc}

% Set font to latin modern
\usepackage{lmodern}

\usepackage[pdftex]{graphicx}
\usepackage{float}
\usepackage{epstopdf}
\usepackage{caption}
\usepackage{subcaption}

% Nice source code listings
\usepackage[usenames,dvipsnames]{xcolor}
\usepackage{listings}
\DeclareCaptionFont{white}{\color{white}}
\DeclareCaptionFormat{listing}{\colorbox{gray}{\parbox{\textwidth}{#1#2#3}}}
\captionsetup[lstlisting]{format=listing,labelfont=white,textfont=white}
\lstset{ %
  backgroundcolor=\color{white},  % choose the background color; you must add \usepackage{color} or \usepackage{xcolor}
  basicstyle=\footnotesize,       % the size of the fonts that are used for the code
  breakatwhitespace=false,        % sets if automatic breaks should only happen at whitespace
  breaklines=true,                % sets automatic line breaking
  numbers=none,                   % where to put the line-numbers; possible values are (none, left, right)
  showspaces=false,               % show spaces everywhere adding particular underscores; it overrides 'showstringspaces'
  showstringspaces=false,         % underline spaces within strings only
  showtabs=false,                 % show tabs within strings adding particular underscores
  tabsize=2,                      % sets default tabsize to 2 spaces
}

% Create links in pdf documents
\usepackage[colorlinks,pdfpagelabels,pdfstartview=FitH,bookmarksopen=true,bookmarksnumbered=true,linkcolor=black,plainpages=false,hypertexnames=false,citecolor=black] {hyperref}
\hypersetup{
    colorlinks,%
    citecolor=black,%
    filecolor=black,%
    linkcolor=black,%
    urlcolor=black
}
\urlstyle{same}

% Use \enquote{} to create quotation marks
\usepackage{csquotes}

% Create professional tables with booktabs
% @see http://en.wikibooks.org/wiki/LaTeX/Tables#Professional_tables
\usepackage{booktabs}

% Customizable enumerates/itemizes
\usepackage{enumitem}

% git meta information
\usepackage{gitinfo}


\usepackage[utf8]{inputenc}
\usepackage[T1]{fontenc}

% Set font to latin modern
\usepackage{lmodern}

\usepackage[pdftex]{graphicx}
\usepackage{float}
\usepackage{epstopdf}
\usepackage{caption}
\usepackage{subcaption}

% Nice source code listings
\usepackage[usenames,dvipsnames]{xcolor}
\usepackage{listings}
\DeclareCaptionFont{white}{\color{white}}
\DeclareCaptionFormat{listing}{\colorbox{gray}{\parbox{\textwidth}{#1#2#3}}}
\captionsetup[lstlisting]{format=listing,labelfont=white,textfont=white}
\lstset{ %
  backgroundcolor=\color{white},  % choose the background color; you must add \usepackage{color} or \usepackage{xcolor}
  basicstyle=\footnotesize,       % the size of the fonts that are used for the code
  breakatwhitespace=false,        % sets if automatic breaks should only happen at whitespace
  breaklines=true,                % sets automatic line breaking
  numbers=none,                   % where to put the line-numbers; possible values are (none, left, right)
  showspaces=false,               % show spaces everywhere adding particular underscores; it overrides 'showstringspaces'
  showstringspaces=false,         % underline spaces within strings only
  showtabs=false,                 % show tabs within strings adding particular underscores
  tabsize=2,                      % sets default tabsize to 2 spaces
}

% Create links in pdf documents
\usepackage[colorlinks,pdfpagelabels,pdfstartview=FitH,bookmarksopen=true,bookmarksnumbered=true,linkcolor=black,plainpages=false,hypertexnames=false,citecolor=black] {hyperref}
\hypersetup{
    colorlinks,%
    citecolor=black,%
    filecolor=black,%
    linkcolor=black,%
    urlcolor=black
}
\urlstyle{same}

% Use \enquote{} to create quotation marks
\usepackage{csquotes}

% Create professional tables with booktabs
% @see http://en.wikibooks.org/wiki/LaTeX/Tables#Professional_tables
\usepackage{booktabs}

% Customizable enumerates/itemizes
\usepackage{enumitem}

% git meta information
\usepackage{gitinfo}


\begin{document}

% Document title for title.tex
\newcommand{\doctitle}{Supplementary Specification}
% Titlepage Template
% ==================
% 
% @author burgc5
% 
% To use this simply enter: % Titlepage Template
% ==================
% 
% @author burgc5
% 
% To use this simply enter: % Titlepage Template
% ==================
% 
% @author burgc5
% 
% To use this simply enter: \input{./title.tex}
% 
% You have to define the commands '\doctitle' and '\docrevision' to give the 
% document a title and a revision on its titlepage.
% Do this with the following command:
% \newcommand{\doctitle}{Document title goes here}
%
% SVN:
% ----
% You also have to define the variables:
% \SVN $Date$
% \SVN $Revision$
%
% As executing the following command on the file:
% > svn propset svn:keywords "Date Revision" filename.tex
% 
% This titlepage needs:
% \usepackage[pdftex]{graphicx}
% \usepackage{svn}
%

\begin{titlepage}

\begin{center}

% Team-logo
\includegraphics[width=0.35\textwidth]{./comet-logo.eps}\\[2.5cm]    

% Project title
\textsc{\Large Pinball Simulator}\\[2cm]

% Document title
{ \huge \bfseries \doctitle{}}\\[3cm]

% Members/Client
\begin{minipage}{0.45\textwidth}
\begin{flushleft} \large
\emph{Team Members:}\\
Patrick \textsc{Haring}\\
Christian \textsc{Bürgi}
\end{flushleft}
\end{minipage}
\begin{minipage}{0.45\textwidth}
\begin{flushright} \large
\emph{Client:} \\
Jean-Pierre \textsc{Caillot}\\
~
\end{flushright}
\end{minipage}

\vfill

{\large 
Revision hash: \gitAbbrevHash \\[0.2cm]
Commit time: \gitCommitterIsoDate \\[0.2cm]
{\footnotesize \itshape \url{https://github.com/boskoop/comet-pinball/blob/master/doc/src/\jobname.tex}}}

\end{center}

\end{titlepage}
% 
% You have to define the commands '\doctitle' and '\docrevision' to give the 
% document a title and a revision on its titlepage.
% Do this with the following command:
% \newcommand{\doctitle}{Document title goes here}
%
% SVN:
% ----
% You also have to define the variables:
% \SVN $Date$
% \SVN $Revision$
%
% As executing the following command on the file:
% > svn propset svn:keywords "Date Revision" filename.tex
% 
% This titlepage needs:
% \usepackage[pdftex]{graphicx}
% \usepackage{svn}
%

\begin{titlepage}

\begin{center}

% Team-logo
\includegraphics[width=0.35\textwidth]{./comet-logo.eps}\\[2.5cm]    

% Project title
\textsc{\Large Pinball Simulator}\\[2cm]

% Document title
{ \huge \bfseries \doctitle{}}\\[3cm]

% Members/Client
\begin{minipage}{0.45\textwidth}
\begin{flushleft} \large
\emph{Team Members:}\\
Patrick \textsc{Haring}\\
Christian \textsc{Bürgi}
\end{flushleft}
\end{minipage}
\begin{minipage}{0.45\textwidth}
\begin{flushright} \large
\emph{Client:} \\
Jean-Pierre \textsc{Caillot}\\
~
\end{flushright}
\end{minipage}

\vfill

{\large 
Revision hash: \gitAbbrevHash \\[0.2cm]
Commit time: \gitCommitterIsoDate \\[0.2cm]
{\footnotesize \itshape \url{https://github.com/boskoop/comet-pinball/blob/master/doc/src/\jobname.tex}}}

\end{center}

\end{titlepage}
% 
% You have to define the commands '\doctitle' and '\docrevision' to give the 
% document a title and a revision on its titlepage.
% Do this with the following command:
% \newcommand{\doctitle}{Document title goes here}
%
% SVN:
% ----
% You also have to define the variables:
% \SVN $Date$
% \SVN $Revision$
%
% As executing the following command on the file:
% > svn propset svn:keywords "Date Revision" filename.tex
% 
% This titlepage needs:
% \usepackage[pdftex]{graphicx}
% \usepackage{svn}
%

\begin{titlepage}

\begin{center}

% Team-logo
\includegraphics[width=0.35\textwidth]{./comet-logo.eps}\\[2.5cm]    

% Project title
\textsc{\Large Pinball Simulator}\\[2cm]

% Document title
{ \huge \bfseries \doctitle{}}\\[3cm]

% Members/Client
\begin{minipage}{0.45\textwidth}
\begin{flushleft} \large
\emph{Team Members:}\\
Patrick \textsc{Haring}\\
Christian \textsc{Bürgi}
\end{flushleft}
\end{minipage}
\begin{minipage}{0.45\textwidth}
\begin{flushright} \large
\emph{Client:} \\
Jean-Pierre \textsc{Caillot}\\
~
\end{flushright}
\end{minipage}

\vfill

{\large 
Revision hash: \gitAbbrevHash \\[0.2cm]
Commit time: \gitCommitterIsoDate \\[0.2cm]
{\footnotesize \itshape \url{https://github.com/boskoop/comet-pinball/blob/master/doc/src/\jobname.tex}}}

\end{center}

\end{titlepage}

\tableofcontents

\section{Introduction}
In Unified Process methodology, the document \emph{Supplementary Specification} contains all the non-functional requirements. (The functional requirements are specified with \emph{Use C	ases}.)

Some additional non-functional requirements are currently covered in the Vision Document.

\section{Non-functional requirements}

\subsection{Usability}
\begin{itemize}
\item[U1] Avoid colors associated with common forms of color blindness.
\item[U2] The interface should be as intuitive as possible.
\item[U3] The key mapping should be customizable.
\end{itemize}


\subsection{Reliability}	

\subsection{Performance}

\begin{itemize}
\item[P1] The application should run on entry-level computer with over 30 frames per second.
\end{itemize}

\subsection{Supportability}

\begin{itemize}
	\item[S1] The system has to run on Unix/GNU-Linux as well as on Microsoft Windows and Mac OS X.	
	\item[S2] The system will be implemented in Java 5.
	\item[S3] The system will be implemented using the cross-platform game development library libgdx http://code.google.com/p/libgdx/. With this library a port for Android or HTML5 would be possible.
\end{itemize}

\end{document}
